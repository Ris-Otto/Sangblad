\documentclass[12pt]{article}

\usepackage[margin=0.4in, bmargin=0.65in, tmargin=0.3in]{geometry}
\usepackage{lexend}
\usepackage{layout}

%Center title on page
\usepackage{titling}
\renewcommand\maketitlehooka{\null\mbox{}\vfill}
\renewcommand\maketitlehookd{\vfill\null}

\title{%
  Sång och blad }

\author{Nocturne no.23 in B-sharp junior 'Choral', op.73}

\date{}


\begin{document}\layout


\maketitle
%remove page number from title page
\thispagestyle{empty}
%hehe numrering 
\renewcommand\thepage{%
\ifcase\value{page}%
43\or
87\or
16\or
7\or
42\or
33\or
103\or
71\or
17\else

\arabic{page}%
\fi}


\newpage
\noindent
\begin{minipage}{.475\textwidth}
	\noindent
	\paragraph*{Dumbom\\}
	\vspace{3px}
	\textit{Mel: egen (wow dehä hjälper mycke!)}\\
\end{minipage}%
\hspace{0.075\textwidth}
\noindent
\begin{minipage}{.45\textwidth}
	\noindent
	\paragraph*{Pyttipanna\\} 
	\vspace{3px}
	\textit{Mel: Godtycklig}\\
\end{minipage}

\noindent
\begin{minipage}{.475\textwidth}
	\noindent
	Ja gör bort mig stup i kvarten\\ hjärnan snubblar ren i starten,\\
	om ja riktigt skärper mig e ja en medelmåtta\\
	Tankeverksamheten sölar\\ för ja saknar geniknölar,\\
	i varje mening lyckas jag min dumhet blotta\\
	För om ja säger si\\
	e svaret alltid så\\
	å hur ja än försöker e ja ute i de blå!\\

	\noindent
	Ja e en dumbom,\\
	mer än lovligt korkad,\\
	för min okunskap blir jag alltid morkkad\\
	Ja e en trögboll, en jubelidiot\\
	på sånt e de svårt att råda bot\\

	\noindent
	Ja har svaret ren på läppen,\\ bara lidnerska knäppen\\
	ville infinna sig\\ å göra mig till ett geni.\\
	Om de hände ett mirakel,\\ å jag blev ett orakel,\\
	en professor i Fiffikus, Nonsens, Struntlogi\\
	Men om ja säger bä\\
	e svaret allti bu\\
	min skonummer e större än min futtiga IQ!\\

	Ja e en...\\
	
	\noindent
	Men i Sju Dvärgarna och Snövit,\\
	ni minns väl hur bedrövligt,\\
	det var för honom som fick bära namnet 'Kloker'\\
	Ja han måste alltid veta,\\
	stressa på å sträta,\\
	å ändå tycker alla mest om lilla Toker\\
	Å ja e just som han\\
	Gör så gott ja kan\\
	Att föla de e mänskligt, den frasen e så sann!\\
	
	Ja e en dumbom,\\
	mer än lovligt korkad,\\
	för min okunskap blir jag alltid morkkad\\
	Ja e en trögboll, ingen krutuppfinnare\\
	men ensam e ja ändå int om de\\
	Men ensam e ja ändå int om de, olé!
\end{minipage}%
\hspace{0.075\textwidth}
\noindent
\begin{minipage}{.45\textwidth}
	\noindent
	Man tager vad man haver,\\
	de (e) regel nummer ett\\
	Man slänger allt i pannan,\\
	å dränker de i fett\\
	Sen knäcker man ett ägg,\\
	å steker rätt å slätt\\
	Här har du ett recept,\\
	som aldrig kan gå snett\\
	
	\noindent
	De (e) lika fint som kaviar\\
	å biff, modell: rost.\\
	De (e) ädelt,\\
	de (e) världsberömt\\
	de (e) husmanskost!\\
	
	\noindent
	Pyttipanna, pyttipanna\\
	kan du få fast i Havanna\\
	eller i Japan i en hamn,\\
	men då har de annat namn.\\
	Pyttipanna, pyttipanna\\
	å de vattnas i din mun! \textbf{\small{*slurp!*}}\\
	De får blickarna att stanna,\\
	på människa, gris å hund!\\
	Pyttipanna!\\
	Pyttipanna!\\
	Pyttipanna!\\
	Delikatess! Delikatess!\\
	Delikatess, oh yes, oh yes!\\
	
	\noindent
	De kungens favoritmat,\\
	i slottet viskas så.\\
	När gästerna har gått sin väg,\\
	ja va händer då?\\
	Kungen smyger ut till köket,\\
	å låser in sig där.\\
	genom nyckelhålet\\
	kan man sen får höra allt det här:\\
	
	\noindent
	Ja e trött på hummer\\
	ja äter de om ja måst\\
	Men tacka vet jag ändå en hederlig\\
	husmanskost!\\
	
	\noindent
	|: Pyttipanna...
	\vspace{1cm}

\end{minipage}


\newpage
\noindent
\begin{minipage}{.5\textwidth}
	\paragraph*{Helikotipetter\\}
	\vspace{3px}
	\textit{Mel: Hjulen på bussen}\\
\end{minipage}%
\hspace{0.05\textwidth}
\noindent
\begin{minipage}{0.45\textwidth}
	\paragraph*{Tambursdraken\\}
	\vspace{3px}
	\textit{Interior: tambur, mel: drake}\\
\end{minipage}

\noindent
\begin{minipage}{.5\textwidth}
	\noindent
	Vi åker helikotipetter,\\
	tillsammans du å jag.\\
	De strittar å de skvätter,\\
	omkring oss när vi far.\\
	Håll i nu hattens brätter,\\
	de sprätter när vi drar.\\
	Vi åker helikotipetter nu\\
	vi åker helikotipetter bra\\
	vi åker helikotipetter du å ja!\\
	Helikoti, helikoti, helikoti, \\
	helikoti, helikoti, helikotihej!\\

	\noindent
	Morsans vispelmotor,\\
	köksfläkten fick bli rotor.\\
	Bord å stol å pallar,\\
	bastutrallar to vi me.\\
	Med koptern ska vi sopa,\\
	genom molnen till Europa,\\
	vi ska loopa, ro och hopa\\
	ut i världen ska ni se!\\
	De går uppåt de går neråt!\\
	De går rättåt de går felåt!\\
	Kompassen en konservburk gjord i plåt! \\

	Vi åker helikotipetter...\\
	
	(\textit{Sångledarpajaskalas - 10 sek paus för\\ försvamling})\\
	
	\noindent
	Där fo köksgardinen,\\
	adjöss till diskmaskinen.\\
	"Kristallkrona i sikte!"\\
	de blev dyrt.\\
	De blev för svåra svängar,\\
	våra sista veckopengar,\\
	får städerskan å firman\\
	där vi tv:n hade hyrt!\\
	Våra busiga bravader\\
	gör att morsan nog får spader!\\
	För i morgon har vi tänkt oss nya hyss! \\

	Vi åker helikotipetter...\\

	\noindent

	|: Helikoti, helikoti, helikoti, \\
	helikoti, helikoti, helikotihej!\\
	(\#, allt mindre självsäkert) :|\\
	(\textit{ad nauseum})
	\vspace{0.2cm}
	
\end{minipage}%
\hspace{0.05\textwidth}
\noindent
\begin{minipage}{0.45\textwidth}
	\noindent
	Vi har en drake i vår tambur!\\
	Bland kapporna ligger den på lur\\
	å låtsas sova ja vet va den gör!\\
	Idag e den kanske på \{sock\}humör\\

	\noindent
	Var e min socka å mina vantar?\\
	Mamma klär på mig men ja bara fjantar.\\
	Sockan e borta å vantarna me,\\
	i drakens mage slank di ner!\\

	Vi har en drake... \{stövel\}\\

	\noindent
	Pappa min stövel, e ingenstans!\\
	Ja såg just skymten av drakens svans.\\
	Stöveln e borta å halsduken me,\\
	i drakens mage slank di ner!\\
	
	\noindent	
	Vi har en drake... \{väsk\}\\
	
	\noindent
	Var e min väska ja ställde den här\\
	\textbf{"Sök den!"} ja söker inte här, inte där.\\
	Väskan e borta å läxorna me,\\
	i drakens mage slank di ner!\\

	\noindent
	Var e min tidning, å mina tofflor?\\
	Allting försvinner, välling å våfflor.\\
	Pipan e borta, å kalsarna me,\\
	i drakens mage slank di ner!\\

	Vi har en drake... \{*pruttljud*\}\\

	Lalalala...\\
	
	\paragraph*{Turturduvan Arthur\\}
	\vspace{3px}
	\textit{Mel: Berg i Turkiet}\\
	
	Turturduvan Arthur\\
	e både klipsk och smarthur!\\
	Med kompass och karthur,\\
	han hittar alltid hem!\\
	
	
	Men en turturduva\\
	kan ibland få snuva\\
	då e de dags att ruva\\
	fram en ny refräng!\\
	%\vspace{5.7cm}
\end{minipage}


\newpage


\noindent
\begin{minipage}{0.45\textwidth}
	\paragraph*{Kottar i karburatorn\\}
	\vspace{3px}
	\textit{Mel: Kottbränsle}\\
\end{minipage}

\noindent
\begin{minipage}{0.45\textwidth}
	\noindent
	Röd man säger "Stop där, stanna!",\\
	grön man säger "Gå!"\\
	Dessa gubbar granna måst du hålla reda på!\\
	
	\noindent
	Va ska vi göra med Pelle?\\
	Titta hur han kör\\
	med sitt gamla cykelskrälle\\
	å ser sig inte för!\\
	Kattögat e borta\\
	å lyktan lyser svart!\\
	Hjälmen har han hemma,\\
	Han tror sig vara smart!\\
	
	\noindent
	Ja får myror i huve av såna som busar på gator!\\
	Di sätter käppar i hjulen å kottar i karburatorn!\\

	\noindent
	Röd man säger "Stop där, stanna!",\\
	grön man säger "Gå!"\\
	Dessa gubbar granna måst du hålla reda på!\\
	Se till vänster, se till höger,\\
	ingen plats för lek!\\
	Där du dansar fram uti\\
	trafikens diskotek!\\
	
	
	\noindent
	Va ska vi göra med Ellen?\\
	Tänk att hon ens täcks\\
	gå i mörka kvällen\\
	utan nån reflex!\\
	Ut bland alla bilar\\
	som bromsar i panik!\\
	De viktigt att man syns\\
	när man traskar i trafik!\\

	\noindent
	Jå ja blir no rikti förbajskad på såna som busar på gator,\\
	Dom sätter ju käppar i hjulen å kottar i karburatorn!\\
	
	Röd man säger...\\
\end{minipage}%
\hspace{0.1\textwidth}
\noindent
\begin{minipage}{0.45\textwidth}
	
	\noindent
	När Pelle å Ellen lär sig\\
	trafikens ABC\\
	vet de hur man bär sig\\
	åt, vi tar dem med!\\

	\noindent
	Jå de no bra riki prima när folk kan bete sig på gator,\\
	de blir ju inga käppar i hjulen å kottar i karburatorn!\\

	Röd man säger...\\

	
	\paragraph*{Hobbel bobbel\\}
	\vspace{3px}
	\textit{Mel: Rått ägg}\\
	
	\noindent
	Min yngre bror som nu e stor,\\
	Han var helt hopplös om du tror.\\
	Sin mat han vägrade äta,\\
	hur mamma än med skeden sträta!\\
	Munnen knep han hastigt fast,\\
	när de var frukost eller middagdags\\
	han sa:\\
	
	\noindent
	Ja vill ha hobbel bobbel i badet,\\
	å ett bad me många bubblor i!\\
	Ja älskar att hobbla å bobbla,\\
	men att äta de låter ja bli!\\
	Ja e familjens krångelbytta,
	nästan allting får mig att spytta\\
	utom hobbel bobbel\\
	i ett baa - aa - ad!\\
	\vspace{5.35cm}
	
\end{minipage}

\newpage

\noindent
\begin{minipage}{0.5\textwidth}
\paragraph*{Trygga räkan\\}
\vspace{3px}
\textit{Mel: Tryggare kan ingen vara, lite[ish]}\\
\end{minipage}%
\hspace{0.1\textwidth}
\noindent
\begin{minipage}{0.4\textwidth}
\paragraph*{Lavabränd\\}
\vspace{3px}
\textit{Mel: Mossa (nästan)}\\
\end{minipage}

\noindent
\begin{minipage}{0.5\textwidth}
	
	Vem e de som smutsar ner i alla våra vatten?\\
	Vem e de som bryr sig bara blanka Kattegatten?\\
	I, om Bottenhavet, Östersjön å Finska viken,\\
	snart ser ut som rena rama \\Smuts- \& Avfallsviken.\\
	
	
	Samma gäller också andra hav å oceaner.\\
	Där de kryssar många handelsfartygskaravaner.\\
	Ibland så åker de på grund med last som kol å olja\\
	Sånt e mindre roligt, de tycker ål å kolja!\\
	
	
	Trygga räkan,\\
	å dess framtid,\\
	annars nalkas oss en skamtid.\\
	I rent vatten,\\
	vill den simma,\\
	föröka sig å int försvinna.\\
	Räkning ger dig algebra,\\
	vi behöver vatten varje da.\\
	Trygga räkan,\\
	å utan tvekan,\\
	om du smutsar må du tro\\
	Kakkar du i eget bo?! (lol)\\
	
	
	Ännu kan vi göra nåt å vara efterkloka.\\
	Innan allt som finns i våra vatten hunnit sloka.\\
	Innan räkan kräks å vräks ifrån sitt hem i havet.\\
	dimppa damp å dumpades,\\
	människan tycks int ha vett!\\
	
	
	Trygga räkan...\\
	
	\noindent
	\textit{Optional: ebin bagpipe solo} -> Trygga räkan...
	
	\vspace{3cm}
\end{minipage}%
\hspace{0.1\textwidth}
\noindent
\begin{minipage}{0.4\textwidth}
	\noindent
	Min farfar har en kort stubin,\\
	han e gammal kanonjär,\\
	som börja på ny kula,\\
	när han blev pangsionär!\\
	För si bastubad e hans passion,\\
	å som skjuten ur kannon,\\
	han skyndar redan opp,\\
	på lavens högsta topp!\\
	
	\noindent
	|: Han e lavabränd!\\
	\textbf{\small{lavabränd}}\\
	Han e lavabränd!\\
	\textbf{\small{lavabränd}}\\
	Lavabränd, lavabränd,\\
	lavabränd, lavabränd,\\
	lavabränd! :|\\
	På bastumätaren,\\
	där finns två herremän:\\
	Celsius å Fahrenheit!\\
	Di har en fajt!\\
	\textbf{\small{Celsius å Fahrenheit\\
	di har en fajt!}}\\
	
	\noindent
	Ingen människa kan va där\\
	när farfar öser på!\\
	Stenarna di väser,\\
	å så han pöser då!\\
	Han har badat flera timmar ren\\
	å sprutar svett som en fontän\\
	men ännu vill han kasta,\\
	mera bad å där med basta!\\
	
	|: Han e lavabränd...\\
	
	\noindent
	Röd, ångande å glad,\\
	han kastar mera bad!\\
	Rullar sig i snö,\\
	å äter HK Bleu!\\
	\textbf{\small{Äter HK Bleu!}}\\
	
	Han e lavabränd...\\(\textit{förutan bastumätare})
	
\end{minipage}

\newpage
\noindent
\begin{minipage}{0.55\textwidth}
	\paragraph*{En kanna öl\\}
	\vspace{3px}
	\textit{Mel: Humle}\\
\end{minipage}
\hspace{0.05\textwidth}
\noindent
\begin{minipage}{0.4\textwidth}
	\paragraph*{Punschen går\\}
	\vspace{3px}
	\textit{Mel: punsch}\\
\end{minipage}

\noindent
\begin{minipage}{0.55\textwidth}
	\noindent
	Livet e för kort för slöseri\\
	Se den vackra värld vi fått att leva i!\\
	Allt de goda finns i överflöd\\
	Vem kan tacka nej till fläsk och nybakt bröd?\\
	\textbf{\small{en kanna öl...?}}\\
	
	\noindent
	En kanna öl!\\
	En kanna öl!\\
	Av små saker byggs min lycka upp!\\
	En kappe rovor, lite smör,\\
	nybakt bröd å gott humör!\\
	Vackra kvinnor å en kanna öl!\\
	
	\noindent
	Livet e för kort för dåligt vin\\
	Sådant lämpar sig för fyllehund å -svin!\\
	Man tar för sig av de som e bäst!\\
	Via Dolorosa blir en glädjefest!\\
	\textbf{\small{var vår gäst...?}}\\
	
	|: En kanna öl...\\
	\vspace{1cm}
	
\end{minipage}%
\hspace{0.05\textwidth}
\noindent
\begin{minipage}{0.4\textwidth}
	\noindent
	Punschen går\\
	sjung hoppfaderallanlallanlej!\\
	Punschen går, \\sjung hoppfaderallanlej!\\
	Å den som inte punschen tar\\
	han ej heller punschen får\\
	Punschen går!\\
	
	\noindent
	Sjung hoppfaderallanlej!\\
	Hej!\\
	
	\noindent
	\textit{Alt.}\\
	Punsch, punsch, punsch,\\
	punschpunschpunschpunschpunsch\\
	punschpunschpunsch...\\
	\vspace{3.5cm}
	
\end{minipage}

\noindent
\begin{minipage}{0.55\textwidth}
	\paragraph*{Svettvisan\\}
	\vspace{3px}
	\textit{Mel: Idk probably nånstans ifrån}\\
\end{minipage}%
\hspace{0.05\textwidth}
\noindent
\begin{minipage}{0.4\textwidth}
	\paragraph*{Vem kan koda\\}
	\vspace{3px}
	\textit{Mel: Gissa(!)}\\
\end{minipage}


\noindent
\begin{minipage}{0.55\textwidth}

	\noindent
	De sköna damerna svettas lätt uti barmen,\\
	å blusen fuktas så smått i veck under armen.\\
	så sant som dörren e som en del utav karmen,\\
	så e odören liksom en del utav charmen.\\

	\noindent
	En herres svettning e mera våldsam å ymnig,\\
	trots att nätskjortan hans e luftig å rymlig.\\
	Ja lite svett gör de bara lätt när han joggar,\\
	å vatten finns ju till för att få svalkande groggar.\\

	\noindent
	Den raska herrn å den sköna damen med vecket,\\
	beslöt att pröva på lite svett under täcket.\\
	Men fastän kärleken e båd döver å blinder,\\
	så har den luktsinnet kvar som ställer till hinder.\\

	\noindent
	Så hör än hit, ni som e så heta på gröten,\\
	å som besväras av vad som doftar från fötren.\\
	Löp inte risk att vid sängen lämnas vid kanten,\\
	håll på er skorna så har ni glädje av tanten.\\
\end{minipage}%
\hspace{0.05\textwidth}
\noindent
\begin{minipage}{.4\textwidth}
	\noindent
	Vem kan koda förutan öl?\\
	Vem kan dricka en liter?\\
	Vem kan supa sig kakkafull\\
	utan DaTeiter?\\
	
	\noindent
	Jag kan koda förutan öl\\
	Jag kan dricka en liter\\
	Men ej supa mig kakkafull\\
	utan DaTeiter\\

	\vspace{5.1cm}
\end{minipage}

\newpage

\paragraph*{Grenadiers (temptitle)\\}
\vspace{3px}
\textit{Mel: British Grenadiers}\\

\noindent
\begin{minipage}{.45\textwidth}

	\noindent
	När jag gick på DaTe\\
	i mina gyllne ungdomsår\\
	Av tidens tand ej sliten\\
	orörd av ålderns spår\\
	Och ja kommer minnas länge än\\
	Hur Albin dunka hårt\\

	\noindent
	Med ett dunk, dunk, dunk, dunk,\\
	dunk, dunk\\
	När vi satt i knaslit vårt!\\
	
	\noindent
	Några dar om året\\
	fick vi dricka donkero\\
	Jag blev lite svag i låret\\
	Och föll i Aura å\\
	Och jag kanske tappa minnet\\
	Men Albin dunka hårt\\

	\noindent
	Med ett dunk...\\

	\noindent
	Jag avundas den unga\\
	kommande årens rikedom\\
	Och häpnas på den äldre;\\
	kandidat i ålderdom\\
	Men alla minnas länge än\\
	hur Albin dunka hårt\\

	\noindent
	Med ett dunk...\\

	\noindent
	En klan utan dess like\\
	vi DaTeiter är\\
	I Kansliet vårat rike\\
	finns båd' lycka och misär\\
	Jag i timmar programmerat där\\
	medan Albin dunka hårt\\

	\noindent
	Med ett dunk...\\
	
	När väl åkallas free(me);\\
	jag till tomhet återgå\\
	Vårt system mig återfordra\\
	nullpointer återstå\\
	Om blott ett minne må förbli\\
	dunkar Albin hårt däri!\\
	
	
\end{minipage}%
\hspace{.1\textwidth}
\noindent
\begin{minipage}{.45\textwidth}
	
	Med ett dunk, dunk, dunk, dunk,\\
	dunk, dunk\\
	När jag satt i vårt kansli!\\

	\noindent
	Så låt oss tömma glasen\\
	och krama om en vän\\
	Lägg detta in i databasen\\
	för att minnas länge än\\
	De gångerna ni varit med\\
	och hjärtat dunkat hårt\\

	\noindent
	Med ett dunk, dunk, dunk, dunk,\\
	dunk, dunk\\
	När vi sjungit DaTe:s låt!\\
	
	\textit{Text: Joar Sabel}\\
	
	\noindent
	\paragraph*{Dator vår\\}
	\vspace{3px}
	\textit{Mel(:D): Pater noster}\\	
	
	Dator vår, som är i Centralen.\\
	Helgad vare din Skärm.\\
	Tillkomme din Data.\\
	Ske din vilja, såsom i Minnet\\
	så ock på skrivaren.\\
	
	\noindent
	Vår dagliga lista giv oss idag\\
	Och förlåt oss våra misstag,\\
	oaktat vår brist att för andra lika göra.\\
	Och inled oss inte i ifrågasättbara krav,\\
	utan fräls oss ifrån affärsmän.\\
	
	\noindent
	Ty Företaget är ditt och Makten\\
	och Personalen, i evighet. Enter.\\

	\vspace{4.6cm}
	
	
\end{minipage}
\newpage

\noindent
\begin{minipage}{.45\textwidth}
	\paragraph*{Bankvisan\\}
	\vspace{3px}
	\textit{Mel: pengar}\\
\end{minipage}%
\hspace{0.1\textwidth}
\noindent
\begin{minipage}{.45\textwidth}
	\paragraph*{Hon var så liten\\}
	\vspace{3px}
	\textit{Mel: is}\\
\end{minipage}
\noindent
\begin{minipage}{.45\textwidth}
	
	Alla har vi skulder\\
	å alla har vi skulder\\
	men ingen har sin oskuld kvar!\\
	Å om ni ännu har den\\
	så kommer vi å tar den!\\
	Raj, raj, raj\\
	raraj, raj, raj!\\
\end{minipage}%
\hspace{0.1\textwidth}
\noindent
\begin{minipage}{.45\textwidth}
	Hon var så liten,\\
	hon var så frusen,\\
	hon hade istappar under musen.\\
	
	\noindent
	Han var en stor en,\\
	han var en tung en,\\
	han hade istappar under pungen.\\
\end{minipage}

\noindent
\begin{minipage}{.45\textwidth}
	\paragraph*{Fiskvisan\\}
	\vspace{3px}
	\textit{Mel: Jag går (int) och fiskar}\\
\end{minipage}%
\hspace{0.1\textwidth}
\noindent
\begin{minipage}{.45\textwidth}
	\paragraph*{Kapitalistvisan\\}
	\vspace{3px}
	\textit{Mel: Smörgås}\\
\end{minipage}

\noindent
\begin{minipage}{.45\textwidth}
	En fisk är fjällig utanpå\\
	och full med ben uti.\\
	Den simmar i var sjö och å\\
	och borde där förbli.\\
	
	\noindent
	Men folk som mindre hjärnor fått,\\
	dom äter fisk med sås.\\
	Det smakar som en igelkott\\
	med mera fjäll, förstås.\\
	
	\noindent
	Och går du hem till nån i stan\\
	som kallar fisk för mat.\\
	Då får du stekt i levertran\\
	en nåldyna på fat.\\
	
	\noindent
	En nåldyna är kanske nätt\\
	med små paljetter på.\\
	Men fiskben går ej ner så lätt\\
	och fjällen kostar på.\\
	
	\noindent
	Och var man än i världen går\\
	så är det likadant.\\
	Fisk med ben som borst man får\\
	och det är riktigt sant.\\
	
	\noindent
	Men gud välsigne den som sa,\\
	det gäller dig och mig:\\
	Till fisken skall man brännvin ta,\\
	då viker benen sig!\\
	
\end{minipage}%
\hspace{0.1\textwidth}
\noindent
\begin{minipage}{.45\textwidth}
	Varje kapitalist\\
	tycker att väst e bäst.\\
	För en socialist\\
	e de en tröst\\
	att solen går upp i öst.\\
	
	\noindent
	Att välja e inte lätt.\\
	Det gäller att välja rätt!\\
	Mången i båd syd och nord\\
	väljer smörgåsbord!\\
	
	\noindent
	\begin{minipage}{1\textwidth}
		\paragraph*{En ensam natt\\}
		\vspace{3px}
		\textit{Mel: En kulen natt}\\
		
		\noindent
		En ensam natt natt natt\\
		min slang ja körde\\
		i täckens vågade, vågade, våg\\
		jag mig berörde\\
		Och hur jag sågade, sågade, såg\\
		jag teknologare lågade låg\\
		långt ner i djupetti, petti, petti, pet\\
		en syndavåg jag förutsåg\\		
	\end{minipage}\\
	\vspace{3.6cm}

\end{minipage}

\end{document}